\begin{table}[ht]
\centering
\begin{tabular}{ll}
  \hline
Name & Description \\ 
  \hline
abline & The abline geom adds a line with specified slope and intercept to the plot. \\ 
  area & An area plot is the continuous analog of a stacked bar chart (see geom\_bar), and can be used to show how composition of the whole varies over the range of x. Choosing the order in which different components is stacked is very important, as it becomes increasing hard to see the individual pattern as you move up the stack. \\ 
  bar & The bar geom is used to produce 1d area plots: bar charts for categorical x, and histograms for continuous y. stat\_bin explains the details of these summaries in more detail. In particular, you can use the weight aesthetic to create weighted histograms and barcharts where the height of the bar no longer represent a count of observations, but a sum over some other variable. See the examples for a practical example. \\ 
  bin2d & Add heatmap of 2d bin counts. \\ 
  blank & The blank geom draws nothing, but can be a useful way of ensuring common scales between different plots. \\ 
  boxplot & The upper and lower "hinges" correspond to the first and third quartiles (the 25th and 75th percentiles). This differs slightly from the method used by the boxplot function, and may be apparent with small samples. See boxplot.stats for for more information on how hinge positions are calculated for boxplot. \\ 
  contour & Display contours of a 3d surface in 2d. \\ 
  crossbar & Hollow bar with middle indicated by horizontal line. \\ 
  density & A smooth density estimate calculated by stat\_density. \\ 
  density2d & Perform a 2D kernel density estimatation using kde2d and display the results with contours. \\ 
  dotplot & In a dot plot, the width of a dot corresponds to the bin width (or maximum width, depending on the binning algorithm), and dots are stacked, with each dot representing one observation. \\ 
  errorbar & Error bars. \\ 
  errorbarh & Horizontal error bars \\ 
  freqpoly & Frequency polygon. \\ 
  hex & Hexagon bining. \\ 
  histogram & geom\_histogram is an alias for geom\_bar plus stat\_bin so you will need to look at the documentation for those objects to get more information about the parameters. \\ 
  hline & This geom allows you to annotate the plot with horizontal lines (see geom\_vline and geom\_abline for other types of lines). \\ 
  jitter & The jitter geom is a convenient default for geom\_point with position = 'jitter'. See position\_jitter to see how to adjust amount of jittering. \\ 
  line & Connect observations, ordered by x value. \\ 
  linerange & An interval represented by a vertical line. \\ 
  map & Does not affect position scales. \\ 
  path & Connect observations in original order \\ 
  point & The point geom is used to create scatterplots. \\ 
  pointrange & An interval represented by a vertical line, with a point in the middle. \\ 
  polygon & Polygon, a filled path. \\ 
  quantile & This can be used as a continuous analogue of a geom\_boxplot. \\ 
  raster & This is a special case of geom\_tile where all tiles are the same size. It is implemented highly efficiently using the internal rasterGrob function. \\ 
  rect & 2d rectangles. \\ 
  ribbon & Ribbons, y range with continuous x values. \\ 
  rug & Marginal rug plots. \\ 
  segment & Single line segments. \\ 
  smooth & Add a smoothed conditional mean. \\ 
  step & Connect observations by stairs. \\ 
  text & Textual annotations. \\ 
  tile & Similar to levelplot and image. \\ 
  violin & Violin plot. \\ 
  vline & This geom allows you to annotate the plot with vertical lines (see geom\_hline and geom\_abline for other types of lines. \\ 
   \hline
\end{tabular}
\caption{Geoms in \texttt{ggplot}} 
\label{geoms}
\end{table}
